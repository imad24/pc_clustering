
% Default to the notebook output style

    


% Inherit from the specified cell style.




    
\documentclass[11pt]{article}

    
    
    \usepackage[T1]{fontenc}
    % Nicer default font (+ math font) than Computer Modern for most use cases
    \usepackage{mathpazo}

    % Basic figure setup, for now with no caption control since it's done
    % automatically by Pandoc (which extracts ![](path) syntax from Markdown).
    \usepackage{graphicx}
    % We will generate all images so they have a width \maxwidth. This means
    % that they will get their normal width if they fit onto the page, but
    % are scaled down if they would overflow the margins.
    \makeatletter
    \def\maxwidth{\ifdim\Gin@nat@width>\linewidth\linewidth
    \else\Gin@nat@width\fi}
    \makeatother
    \let\Oldincludegraphics\includegraphics
    % Set max figure width to be 80% of text width, for now hardcoded.
    \renewcommand{\includegraphics}[1]{\Oldincludegraphics[width=.8\maxwidth]{#1}}
    % Ensure that by default, figures have no caption (until we provide a
    % proper Figure object with a Caption API and a way to capture that
    % in the conversion process - todo).
    \usepackage{caption}
    \DeclareCaptionLabelFormat{nolabel}{}
    \captionsetup{labelformat=nolabel}

    \usepackage{adjustbox} % Used to constrain images to a maximum size 
    \usepackage{xcolor} % Allow colors to be defined
    \usepackage{enumerate} % Needed for markdown enumerations to work
    \usepackage{geometry} % Used to adjust the document margins
    \usepackage{amsmath} % Equations
    \usepackage{amssymb} % Equations
    \usepackage{textcomp} % defines textquotesingle
    % Hack from http://tex.stackexchange.com/a/47451/13684:
    \AtBeginDocument{%
        \def\PYZsq{\textquotesingle}% Upright quotes in Pygmentized code
    }
    \usepackage{upquote} % Upright quotes for verbatim code
    \usepackage{eurosym} % defines \euro
    \usepackage[mathletters]{ucs} % Extended unicode (utf-8) support
    \usepackage[utf8x]{inputenc} % Allow utf-8 characters in the tex document
    \usepackage{fancyvrb} % verbatim replacement that allows latex
    \usepackage{grffile} % extends the file name processing of package graphics 
                         % to support a larger range 
    % The hyperref package gives us a pdf with properly built
    % internal navigation ('pdf bookmarks' for the table of contents,
    % internal cross-reference links, web links for URLs, etc.)
    \usepackage{hyperref}
    \usepackage{longtable} % longtable support required by pandoc >1.10
    \usepackage{booktabs}  % table support for pandoc > 1.12.2
    \usepackage[inline]{enumitem} % IRkernel/repr support (it uses the enumerate* environment)
    \usepackage[normalem]{ulem} % ulem is needed to support strikethroughs (\sout)
                                % normalem makes italics be italics, not underlines
    

    
    
    % Colors for the hyperref package
    \definecolor{urlcolor}{rgb}{0,.145,.698}
    \definecolor{linkcolor}{rgb}{.71,0.21,0.01}
    \definecolor{citecolor}{rgb}{.12,.54,.11}

    % ANSI colors
    \definecolor{ansi-black}{HTML}{3E424D}
    \definecolor{ansi-black-intense}{HTML}{282C36}
    \definecolor{ansi-red}{HTML}{E75C58}
    \definecolor{ansi-red-intense}{HTML}{B22B31}
    \definecolor{ansi-green}{HTML}{00A250}
    \definecolor{ansi-green-intense}{HTML}{007427}
    \definecolor{ansi-yellow}{HTML}{DDB62B}
    \definecolor{ansi-yellow-intense}{HTML}{B27D12}
    \definecolor{ansi-blue}{HTML}{208FFB}
    \definecolor{ansi-blue-intense}{HTML}{0065CA}
    \definecolor{ansi-magenta}{HTML}{D160C4}
    \definecolor{ansi-magenta-intense}{HTML}{A03196}
    \definecolor{ansi-cyan}{HTML}{60C6C8}
    \definecolor{ansi-cyan-intense}{HTML}{258F8F}
    \definecolor{ansi-white}{HTML}{C5C1B4}
    \definecolor{ansi-white-intense}{HTML}{A1A6B2}

    % commands and environments needed by pandoc snippets
    % extracted from the output of `pandoc -s`
    \providecommand{\tightlist}{%
      \setlength{\itemsep}{0pt}\setlength{\parskip}{0pt}}
    \DefineVerbatimEnvironment{Highlighting}{Verbatim}{commandchars=\\\{\}}
    % Add ',fontsize=\small' for more characters per line
    \newenvironment{Shaded}{}{}
    \newcommand{\KeywordTok}[1]{\textcolor[rgb]{0.00,0.44,0.13}{\textbf{{#1}}}}
    \newcommand{\DataTypeTok}[1]{\textcolor[rgb]{0.56,0.13,0.00}{{#1}}}
    \newcommand{\DecValTok}[1]{\textcolor[rgb]{0.25,0.63,0.44}{{#1}}}
    \newcommand{\BaseNTok}[1]{\textcolor[rgb]{0.25,0.63,0.44}{{#1}}}
    \newcommand{\FloatTok}[1]{\textcolor[rgb]{0.25,0.63,0.44}{{#1}}}
    \newcommand{\CharTok}[1]{\textcolor[rgb]{0.25,0.44,0.63}{{#1}}}
    \newcommand{\StringTok}[1]{\textcolor[rgb]{0.25,0.44,0.63}{{#1}}}
    \newcommand{\CommentTok}[1]{\textcolor[rgb]{0.38,0.63,0.69}{\textit{{#1}}}}
    \newcommand{\OtherTok}[1]{\textcolor[rgb]{0.00,0.44,0.13}{{#1}}}
    \newcommand{\AlertTok}[1]{\textcolor[rgb]{1.00,0.00,0.00}{\textbf{{#1}}}}
    \newcommand{\FunctionTok}[1]{\textcolor[rgb]{0.02,0.16,0.49}{{#1}}}
    \newcommand{\RegionMarkerTok}[1]{{#1}}
    \newcommand{\ErrorTok}[1]{\textcolor[rgb]{1.00,0.00,0.00}{\textbf{{#1}}}}
    \newcommand{\NormalTok}[1]{{#1}}
    
    % Additional commands for more recent versions of Pandoc
    \newcommand{\ConstantTok}[1]{\textcolor[rgb]{0.53,0.00,0.00}{{#1}}}
    \newcommand{\SpecialCharTok}[1]{\textcolor[rgb]{0.25,0.44,0.63}{{#1}}}
    \newcommand{\VerbatimStringTok}[1]{\textcolor[rgb]{0.25,0.44,0.63}{{#1}}}
    \newcommand{\SpecialStringTok}[1]{\textcolor[rgb]{0.73,0.40,0.53}{{#1}}}
    \newcommand{\ImportTok}[1]{{#1}}
    \newcommand{\DocumentationTok}[1]{\textcolor[rgb]{0.73,0.13,0.13}{\textit{{#1}}}}
    \newcommand{\AnnotationTok}[1]{\textcolor[rgb]{0.38,0.63,0.69}{\textbf{\textit{{#1}}}}}
    \newcommand{\CommentVarTok}[1]{\textcolor[rgb]{0.38,0.63,0.69}{\textbf{\textit{{#1}}}}}
    \newcommand{\VariableTok}[1]{\textcolor[rgb]{0.10,0.09,0.49}{{#1}}}
    \newcommand{\ControlFlowTok}[1]{\textcolor[rgb]{0.00,0.44,0.13}{\textbf{{#1}}}}
    \newcommand{\OperatorTok}[1]{\textcolor[rgb]{0.40,0.40,0.40}{{#1}}}
    \newcommand{\BuiltInTok}[1]{{#1}}
    \newcommand{\ExtensionTok}[1]{{#1}}
    \newcommand{\PreprocessorTok}[1]{\textcolor[rgb]{0.74,0.48,0.00}{{#1}}}
    \newcommand{\AttributeTok}[1]{\textcolor[rgb]{0.49,0.56,0.16}{{#1}}}
    \newcommand{\InformationTok}[1]{\textcolor[rgb]{0.38,0.63,0.69}{\textbf{\textit{{#1}}}}}
    \newcommand{\WarningTok}[1]{\textcolor[rgb]{0.38,0.63,0.69}{\textbf{\textit{{#1}}}}}
    
    
    % Define a nice break command that doesn't care if a line doesn't already
    % exist.
    \def\br{\hspace*{\fill} \\* }
    % Math Jax compatability definitions
    \def\gt{>}
    \def\lt{<}
    % Document parameters
    \title{Report}
    
    
    

    % Pygments definitions
    
\makeatletter
\def\PY@reset{\let\PY@it=\relax \let\PY@bf=\relax%
    \let\PY@ul=\relax \let\PY@tc=\relax%
    \let\PY@bc=\relax \let\PY@ff=\relax}
\def\PY@tok#1{\csname PY@tok@#1\endcsname}
\def\PY@toks#1+{\ifx\relax#1\empty\else%
    \PY@tok{#1}\expandafter\PY@toks\fi}
\def\PY@do#1{\PY@bc{\PY@tc{\PY@ul{%
    \PY@it{\PY@bf{\PY@ff{#1}}}}}}}
\def\PY#1#2{\PY@reset\PY@toks#1+\relax+\PY@do{#2}}

\expandafter\def\csname PY@tok@w\endcsname{\def\PY@tc##1{\textcolor[rgb]{0.73,0.73,0.73}{##1}}}
\expandafter\def\csname PY@tok@c\endcsname{\let\PY@it=\textit\def\PY@tc##1{\textcolor[rgb]{0.25,0.50,0.50}{##1}}}
\expandafter\def\csname PY@tok@cp\endcsname{\def\PY@tc##1{\textcolor[rgb]{0.74,0.48,0.00}{##1}}}
\expandafter\def\csname PY@tok@k\endcsname{\let\PY@bf=\textbf\def\PY@tc##1{\textcolor[rgb]{0.00,0.50,0.00}{##1}}}
\expandafter\def\csname PY@tok@kp\endcsname{\def\PY@tc##1{\textcolor[rgb]{0.00,0.50,0.00}{##1}}}
\expandafter\def\csname PY@tok@kt\endcsname{\def\PY@tc##1{\textcolor[rgb]{0.69,0.00,0.25}{##1}}}
\expandafter\def\csname PY@tok@o\endcsname{\def\PY@tc##1{\textcolor[rgb]{0.40,0.40,0.40}{##1}}}
\expandafter\def\csname PY@tok@ow\endcsname{\let\PY@bf=\textbf\def\PY@tc##1{\textcolor[rgb]{0.67,0.13,1.00}{##1}}}
\expandafter\def\csname PY@tok@nb\endcsname{\def\PY@tc##1{\textcolor[rgb]{0.00,0.50,0.00}{##1}}}
\expandafter\def\csname PY@tok@nf\endcsname{\def\PY@tc##1{\textcolor[rgb]{0.00,0.00,1.00}{##1}}}
\expandafter\def\csname PY@tok@nc\endcsname{\let\PY@bf=\textbf\def\PY@tc##1{\textcolor[rgb]{0.00,0.00,1.00}{##1}}}
\expandafter\def\csname PY@tok@nn\endcsname{\let\PY@bf=\textbf\def\PY@tc##1{\textcolor[rgb]{0.00,0.00,1.00}{##1}}}
\expandafter\def\csname PY@tok@ne\endcsname{\let\PY@bf=\textbf\def\PY@tc##1{\textcolor[rgb]{0.82,0.25,0.23}{##1}}}
\expandafter\def\csname PY@tok@nv\endcsname{\def\PY@tc##1{\textcolor[rgb]{0.10,0.09,0.49}{##1}}}
\expandafter\def\csname PY@tok@no\endcsname{\def\PY@tc##1{\textcolor[rgb]{0.53,0.00,0.00}{##1}}}
\expandafter\def\csname PY@tok@nl\endcsname{\def\PY@tc##1{\textcolor[rgb]{0.63,0.63,0.00}{##1}}}
\expandafter\def\csname PY@tok@ni\endcsname{\let\PY@bf=\textbf\def\PY@tc##1{\textcolor[rgb]{0.60,0.60,0.60}{##1}}}
\expandafter\def\csname PY@tok@na\endcsname{\def\PY@tc##1{\textcolor[rgb]{0.49,0.56,0.16}{##1}}}
\expandafter\def\csname PY@tok@nt\endcsname{\let\PY@bf=\textbf\def\PY@tc##1{\textcolor[rgb]{0.00,0.50,0.00}{##1}}}
\expandafter\def\csname PY@tok@nd\endcsname{\def\PY@tc##1{\textcolor[rgb]{0.67,0.13,1.00}{##1}}}
\expandafter\def\csname PY@tok@s\endcsname{\def\PY@tc##1{\textcolor[rgb]{0.73,0.13,0.13}{##1}}}
\expandafter\def\csname PY@tok@sd\endcsname{\let\PY@it=\textit\def\PY@tc##1{\textcolor[rgb]{0.73,0.13,0.13}{##1}}}
\expandafter\def\csname PY@tok@si\endcsname{\let\PY@bf=\textbf\def\PY@tc##1{\textcolor[rgb]{0.73,0.40,0.53}{##1}}}
\expandafter\def\csname PY@tok@se\endcsname{\let\PY@bf=\textbf\def\PY@tc##1{\textcolor[rgb]{0.73,0.40,0.13}{##1}}}
\expandafter\def\csname PY@tok@sr\endcsname{\def\PY@tc##1{\textcolor[rgb]{0.73,0.40,0.53}{##1}}}
\expandafter\def\csname PY@tok@ss\endcsname{\def\PY@tc##1{\textcolor[rgb]{0.10,0.09,0.49}{##1}}}
\expandafter\def\csname PY@tok@sx\endcsname{\def\PY@tc##1{\textcolor[rgb]{0.00,0.50,0.00}{##1}}}
\expandafter\def\csname PY@tok@m\endcsname{\def\PY@tc##1{\textcolor[rgb]{0.40,0.40,0.40}{##1}}}
\expandafter\def\csname PY@tok@gh\endcsname{\let\PY@bf=\textbf\def\PY@tc##1{\textcolor[rgb]{0.00,0.00,0.50}{##1}}}
\expandafter\def\csname PY@tok@gu\endcsname{\let\PY@bf=\textbf\def\PY@tc##1{\textcolor[rgb]{0.50,0.00,0.50}{##1}}}
\expandafter\def\csname PY@tok@gd\endcsname{\def\PY@tc##1{\textcolor[rgb]{0.63,0.00,0.00}{##1}}}
\expandafter\def\csname PY@tok@gi\endcsname{\def\PY@tc##1{\textcolor[rgb]{0.00,0.63,0.00}{##1}}}
\expandafter\def\csname PY@tok@gr\endcsname{\def\PY@tc##1{\textcolor[rgb]{1.00,0.00,0.00}{##1}}}
\expandafter\def\csname PY@tok@ge\endcsname{\let\PY@it=\textit}
\expandafter\def\csname PY@tok@gs\endcsname{\let\PY@bf=\textbf}
\expandafter\def\csname PY@tok@gp\endcsname{\let\PY@bf=\textbf\def\PY@tc##1{\textcolor[rgb]{0.00,0.00,0.50}{##1}}}
\expandafter\def\csname PY@tok@go\endcsname{\def\PY@tc##1{\textcolor[rgb]{0.53,0.53,0.53}{##1}}}
\expandafter\def\csname PY@tok@gt\endcsname{\def\PY@tc##1{\textcolor[rgb]{0.00,0.27,0.87}{##1}}}
\expandafter\def\csname PY@tok@err\endcsname{\def\PY@bc##1{\setlength{\fboxsep}{0pt}\fcolorbox[rgb]{1.00,0.00,0.00}{1,1,1}{\strut ##1}}}
\expandafter\def\csname PY@tok@kc\endcsname{\let\PY@bf=\textbf\def\PY@tc##1{\textcolor[rgb]{0.00,0.50,0.00}{##1}}}
\expandafter\def\csname PY@tok@kd\endcsname{\let\PY@bf=\textbf\def\PY@tc##1{\textcolor[rgb]{0.00,0.50,0.00}{##1}}}
\expandafter\def\csname PY@tok@kn\endcsname{\let\PY@bf=\textbf\def\PY@tc##1{\textcolor[rgb]{0.00,0.50,0.00}{##1}}}
\expandafter\def\csname PY@tok@kr\endcsname{\let\PY@bf=\textbf\def\PY@tc##1{\textcolor[rgb]{0.00,0.50,0.00}{##1}}}
\expandafter\def\csname PY@tok@bp\endcsname{\def\PY@tc##1{\textcolor[rgb]{0.00,0.50,0.00}{##1}}}
\expandafter\def\csname PY@tok@fm\endcsname{\def\PY@tc##1{\textcolor[rgb]{0.00,0.00,1.00}{##1}}}
\expandafter\def\csname PY@tok@vc\endcsname{\def\PY@tc##1{\textcolor[rgb]{0.10,0.09,0.49}{##1}}}
\expandafter\def\csname PY@tok@vg\endcsname{\def\PY@tc##1{\textcolor[rgb]{0.10,0.09,0.49}{##1}}}
\expandafter\def\csname PY@tok@vi\endcsname{\def\PY@tc##1{\textcolor[rgb]{0.10,0.09,0.49}{##1}}}
\expandafter\def\csname PY@tok@vm\endcsname{\def\PY@tc##1{\textcolor[rgb]{0.10,0.09,0.49}{##1}}}
\expandafter\def\csname PY@tok@sa\endcsname{\def\PY@tc##1{\textcolor[rgb]{0.73,0.13,0.13}{##1}}}
\expandafter\def\csname PY@tok@sb\endcsname{\def\PY@tc##1{\textcolor[rgb]{0.73,0.13,0.13}{##1}}}
\expandafter\def\csname PY@tok@sc\endcsname{\def\PY@tc##1{\textcolor[rgb]{0.73,0.13,0.13}{##1}}}
\expandafter\def\csname PY@tok@dl\endcsname{\def\PY@tc##1{\textcolor[rgb]{0.73,0.13,0.13}{##1}}}
\expandafter\def\csname PY@tok@s2\endcsname{\def\PY@tc##1{\textcolor[rgb]{0.73,0.13,0.13}{##1}}}
\expandafter\def\csname PY@tok@sh\endcsname{\def\PY@tc##1{\textcolor[rgb]{0.73,0.13,0.13}{##1}}}
\expandafter\def\csname PY@tok@s1\endcsname{\def\PY@tc##1{\textcolor[rgb]{0.73,0.13,0.13}{##1}}}
\expandafter\def\csname PY@tok@mb\endcsname{\def\PY@tc##1{\textcolor[rgb]{0.40,0.40,0.40}{##1}}}
\expandafter\def\csname PY@tok@mf\endcsname{\def\PY@tc##1{\textcolor[rgb]{0.40,0.40,0.40}{##1}}}
\expandafter\def\csname PY@tok@mh\endcsname{\def\PY@tc##1{\textcolor[rgb]{0.40,0.40,0.40}{##1}}}
\expandafter\def\csname PY@tok@mi\endcsname{\def\PY@tc##1{\textcolor[rgb]{0.40,0.40,0.40}{##1}}}
\expandafter\def\csname PY@tok@il\endcsname{\def\PY@tc##1{\textcolor[rgb]{0.40,0.40,0.40}{##1}}}
\expandafter\def\csname PY@tok@mo\endcsname{\def\PY@tc##1{\textcolor[rgb]{0.40,0.40,0.40}{##1}}}
\expandafter\def\csname PY@tok@ch\endcsname{\let\PY@it=\textit\def\PY@tc##1{\textcolor[rgb]{0.25,0.50,0.50}{##1}}}
\expandafter\def\csname PY@tok@cm\endcsname{\let\PY@it=\textit\def\PY@tc##1{\textcolor[rgb]{0.25,0.50,0.50}{##1}}}
\expandafter\def\csname PY@tok@cpf\endcsname{\let\PY@it=\textit\def\PY@tc##1{\textcolor[rgb]{0.25,0.50,0.50}{##1}}}
\expandafter\def\csname PY@tok@c1\endcsname{\let\PY@it=\textit\def\PY@tc##1{\textcolor[rgb]{0.25,0.50,0.50}{##1}}}
\expandafter\def\csname PY@tok@cs\endcsname{\let\PY@it=\textit\def\PY@tc##1{\textcolor[rgb]{0.25,0.50,0.50}{##1}}}

\def\PYZbs{\char`\\}
\def\PYZus{\char`\_}
\def\PYZob{\char`\{}
\def\PYZcb{\char`\}}
\def\PYZca{\char`\^}
\def\PYZam{\char`\&}
\def\PYZlt{\char`\<}
\def\PYZgt{\char`\>}
\def\PYZsh{\char`\#}
\def\PYZpc{\char`\%}
\def\PYZdl{\char`\$}
\def\PYZhy{\char`\-}
\def\PYZsq{\char`\'}
\def\PYZdq{\char`\"}
\def\PYZti{\char`\~}
% for compatibility with earlier versions
\def\PYZat{@}
\def\PYZlb{[}
\def\PYZrb{]}
\makeatother


    % Exact colors from NB
    \definecolor{incolor}{rgb}{0.0, 0.0, 0.5}
    \definecolor{outcolor}{rgb}{0.545, 0.0, 0.0}



    
    % Prevent overflowing lines due to hard-to-break entities
    \sloppy 
    % Setup hyperref package
    \hypersetup{
      breaklinks=true,  % so long urls are correctly broken across lines
      colorlinks=true,
      urlcolor=urlcolor,
      linkcolor=linkcolor,
      citecolor=citecolor,
      }
    % Slightly bigger margins than the latex defaults
    
    \geometry{verbose,tmargin=1in,bmargin=1in,lmargin=1in,rmargin=1in}
    
    

    \begin{document}
    
    
    \maketitle
    
    

    
    \section{1- Data Preparation}\label{data-preparation}

    \subsection{The Data}\label{the-data}

On dispose des données de vente journalière de chez Bonduelle.
L'extraction contient l'historique de ventes entre le premier trimestre
2016 jusqu'à la 3 semaine de janvier 2018.

Les ventes sont groupées au niveau P2 où tous les produits sont pris en
compte séparément et en incluant les ventes promotionnelles. Le
groupement des ventes par client est fait au niveau C4 où les résultats
sont sommés par distributeur.

    \begin{Verbatim}[commandchars=\\\{\}]
{\color{incolor}In [{\color{incolor}20}]:} \PY{n}{product\PYZus{}raw\PYZus{}df}\PY{o}{.}\PY{n}{head}\PY{p}{(}\PY{p}{)}
\end{Verbatim}


\begin{Verbatim}[commandchars=\\\{\}]
{\color{outcolor}Out[{\color{outcolor}20}]:}           Product  Client  04/01/2016  05/01/2016  06/01/2016  07/01/2016  \textbackslash{}
         0  GBA001AUC180FS  68L041        0.00        2.00        0.00        2.00   
         1  GBA001AUC180FS  68L042        0.00        1.00        2.00        1.00   
         2  GBA001BND060FS  68C011        0.00        0.00        0.00        0.00   
         3  GBA001BND060FS  68C120        0.00        0.00        0.00        0.00   
         4  GBA001BND060FS  68C123        0.00        0.00        0.00        0.00   
         
            08/01/2016  09/01/2016  10/01/2016  11/01/2016     {\ldots}      14/01/2018  \textbackslash{}
         0        0.00        3.00        0.00        0.00     {\ldots}            0.00   
         1        0.00        2.00        0.00        1.00     {\ldots}            0.00   
         2        0.00        0.00        0.00        0.00     {\ldots}            0.00   
         3        0.00        0.00        0.00        0.00     {\ldots}            0.00   
         4        0.00        0.00        0.00        0.00     {\ldots}            0.00   
         
            15/01/2018  16/01/2018  17/01/2018  18/01/2018  19/01/2018  20/01/2018  \textbackslash{}
         0        0.00        0.00        0.00        0.00        0.00        0.00   
         1        0.00        0.00        0.00        0.00        0.00        0.00   
         2        0.00        0.00        0.00        0.00        0.00        0.00   
         3        0.00        0.00        0.00        0.00        0.00        0.00   
         4        0.00        0.00        0.00        0.00        0.00        0.00   
         
            21/01/2018  22/01/2018  23/01/2018  
         0        0.00        0.00        0.00  
         1        0.00        0.00        0.00  
         2        0.00        0.00        0.00  
         3        0.00        0.00        0.00  
         4        0.00        0.00        0.00  
         
         [5 rows x 753 columns]
\end{Verbatim}
            
    \subsubsection{Week Resampling}\label{week-resampling}

La fluctuation des ventes journalières de certains produits étant très
importantes, nous allons considérer les ventes sur la base des
historiques hebdomadaires.

    \begin{Verbatim}[commandchars=\\\{\}]
{\color{incolor}In [{\color{incolor}21}]:} \PY{n}{weekly\PYZus{}product\PYZus{}df} \PY{o}{=} \PY{n}{product\PYZus{}df}\PY{o}{.}\PY{n}{resample}\PY{p}{(}\PY{l+s+s1}{\PYZsq{}}\PY{l+s+s1}{W}\PY{l+s+s1}{\PYZsq{}}\PY{p}{,}\PY{n}{axis}\PY{o}{=}\PY{l+m+mi}{1}\PY{p}{)}\PY{o}{.}\PY{n}{mean}\PY{p}{(}\PY{p}{)}
         \PY{n}{product\PYZus{}df}\PY{o}{.}\PY{n}{head}\PY{p}{(}\PY{p}{)}
\end{Verbatim}


\begin{Verbatim}[commandchars=\\\{\}]
{\color{outcolor}Out[{\color{outcolor}21}]:}    2016-01-10  2016-01-17  2016-01-24  2016-01-31  2016-02-07  2016-02-14  \textbackslash{}
         0        1.00        1.14        1.29        1.00        0.71        1.43   
         1        0.86        1.43        1.57        1.57        1.71        1.29   
         2        0.00        0.00        0.00        0.00        0.00        0.00   
         3        0.00        0.00        0.00        0.00        0.00        0.00   
         4        0.00        0.00        0.00        0.00        0.00        0.00   
         
            2016-02-21  2016-02-28  2016-03-06  2016-03-13     {\ldots}      2017-11-26  \textbackslash{}
         0        1.86        1.57        1.00        1.57     {\ldots}            0.00   
         1        1.71        1.29        1.00        0.57     {\ldots}            0.00   
         2        0.00        0.00        0.00        0.00     {\ldots}            0.00   
         3        0.00        0.00        0.00        0.00     {\ldots}            2.29   
         4        0.00        0.00        0.00        0.00     {\ldots}            0.00   
         
            2017-12-03  2017-12-10  2017-12-17  2017-12-24  2017-12-31  2018-01-07  \textbackslash{}
         0        0.00        0.00        0.00        0.00        0.00        0.00   
         1        0.00        0.00        0.00        0.00        0.00        0.00   
         2        0.00        0.00        0.00        0.00        0.00        0.00   
         3        1.29        0.00        0.00        0.00        0.00        0.00   
         4        0.00        0.00        0.00        0.00        0.00        0.00   
         
            2018-01-14  2018-01-21  2018-01-28  
         0        0.00        0.00        0.00  
         1        0.00        0.00        0.00  
         2        0.00        0.00        0.00  
         3        0.00        0.00        0.00  
         4        0.00        0.00        0.00  
         
         [5 rows x 108 columns]
\end{Verbatim}
            
    \section{Cleaning Series}\label{cleaning-series}

A series of cleaning functions is applied to raw data in order to get
rid of irrelevant and "dirty" data with patterns that could disrupt the
model.

Each function has a threashold parameter in order to adapt the filtering

\subsubsection{Trimming Zeros from
series}\label{trimming-zeros-from-series}

Remove empty data on the two sides

    \begin{Verbatim}[commandchars=\\\{\}]
{\color{incolor}In [{\color{incolor}23}]:} \PY{n}{product\PYZus{}df}\PY{o}{.}\PY{n}{head}\PY{p}{(}\PY{p}{)}
\end{Verbatim}


    \begin{Verbatim}[commandchars=\\\{\}]
The last 8 values (complete zeros) of each series have been dropped 
The first 0 values (complete zeros) of each series have been dropped 

    \end{Verbatim}

\begin{Verbatim}[commandchars=\\\{\}]
{\color{outcolor}Out[{\color{outcolor}23}]:}    2016-01-10  2016-01-17  2016-01-24  2016-01-31  2016-02-07  2016-02-14  \textbackslash{}
         0        1.00        1.14        1.29        1.00        0.71        1.43   
         1        0.86        1.43        1.57        1.57        1.71        1.29   
         2        0.00        0.00        0.00        0.00        0.00        0.00   
         3        0.00        0.00        0.00        0.00        0.00        0.00   
         4        0.00        0.00        0.00        0.00        0.00        0.00   
         
            2016-02-21  2016-02-28  2016-03-06  2016-03-13     {\ldots}      2017-10-01  \textbackslash{}
         0        1.86        1.57        1.00        1.57     {\ldots}            0.00   
         1        1.71        1.29        1.00        0.57     {\ldots}            0.00   
         2        0.00        0.00        0.00        0.00     {\ldots}            1.14   
         3        0.00        0.00        0.00        0.00     {\ldots}            1.57   
         4        0.00        0.00        0.00        0.00     {\ldots}            0.00   
         
            2017-10-08  2017-10-15  2017-10-22  2017-10-29  2017-11-05  2017-11-12  \textbackslash{}
         0        0.00        0.00        0.00        0.00        0.00        0.00   
         1        0.00        0.00        0.00        0.00        0.00        0.00   
         2        1.57        1.14        0.86        1.29        1.14        0.29   
         3        2.29        2.00        2.14        2.00        2.43        2.86   
         4        0.00        0.00        0.00        0.00        0.00        0.00   
         
            2017-11-19  2017-11-26  2017-12-03  
         0        0.00        0.00        0.00  
         1        0.00        0.00        0.00  
         2        0.00        0.00        0.00  
         3        1.14        2.29        1.29  
         4        0.00        0.00        0.00  
         
         [5 rows x 100 columns]
\end{Verbatim}
            
    \subsubsection{Tailing Zeros: No longer
sold}\label{tailing-zeros-no-longer-sold}

Remove the products that werent sold in the \textbf{last 15 weeks}

    \begin{Verbatim}[commandchars=\\\{\}]
{\color{incolor}In [{\color{incolor}24}]:} \PY{n}{t} \PY{o}{=} \PY{l+m+mi}{15}
\end{Verbatim}


    \begin{center}
    \adjustimage{max size={0.9\linewidth}{0.9\paperheight}}{output_8_0.png}
    \end{center}
    { \hspace*{\fill} \\}
    
    \begin{Verbatim}[commandchars=\\\{\}]
Series With 15 trailing zeros are removed
Removed: 4327 , Remaining: 4319

    \end{Verbatim}

    \subsubsection{Leading Zeros: Recently
launched}\label{leading-zeros-recently-launched}

Remove the products which werent sold in the \textbf{15 first weeks}

    \begin{Verbatim}[commandchars=\\\{\}]
{\color{incolor}In [{\color{incolor}25}]:} \PY{n}{t} \PY{o}{=} \PY{l+m+mi}{15}
\end{Verbatim}


    \begin{center}
    \adjustimage{max size={0.9\linewidth}{0.9\paperheight}}{output_10_0.png}
    \end{center}
    { \hspace*{\fill} \\}
    
    \begin{Verbatim}[commandchars=\\\{\}]
Series With more than 15 zeros are removed
Removed: 1472 , Remaining: 2847

    \end{Verbatim}

    \begin{Verbatim}[commandchars=\\\{\}]
{\color{incolor}In [{\color{incolor}27}]:} \PY{n}{product\PYZus{}df\PYZus{}clean} \PY{o}{=} \PY{n}{product\PYZus{}df}
         \PY{n+nb}{print}\PY{p}{(}\PY{n}{product\PYZus{}df}\PY{o}{.}\PY{n}{shape}\PY{p}{)}
         \PY{n}{product\PYZus{}df}\PY{o}{.}\PY{n}{head}\PY{p}{(}\PY{p}{)}
\end{Verbatim}


    \begin{Verbatim}[commandchars=\\\{\}]
(2847, 100)

    \end{Verbatim}

\begin{Verbatim}[commandchars=\\\{\}]
{\color{outcolor}Out[{\color{outcolor}27}]:}     2016-01-10  2016-01-17  2016-01-24  2016-01-31  2016-02-07  2016-02-14  \textbackslash{}
         37        4.71        4.29        4.71        5.14        4.14        4.57   
         39        8.57       13.71       11.86        6.29        6.29        6.14   
         40        3.71        0.86        0.43        0.14        0.29        0.29   
         44        1.14        1.29        0.86        1.43        2.86        1.29   
         45        1.14        0.71        0.86        1.14        1.14        1.14   
         
             2016-02-21  2016-02-28  2016-03-06  2016-03-13     {\ldots}      2017-10-01  \textbackslash{}
         37        4.71        5.00        4.00        4.57     {\ldots}            2.29   
         39        6.29        5.14        5.29        6.14     {\ldots}            7.86   
         40        1.00        0.43        2.00        4.14     {\ldots}            1.29   
         44        1.57        1.86        1.86        1.71     {\ldots}            0.29   
         45        1.43        0.86        1.14        0.86     {\ldots}            0.86   
         
             2017-10-08  2017-10-15  2017-10-22  2017-10-29  2017-11-05  2017-11-12  \textbackslash{}
         37        1.71        2.14        1.86        2.29        2.14        1.71   
         39        9.71       19.29       17.29        9.00        8.00        7.14   
         40        1.29        1.29        1.29        1.00        1.29        1.00   
         44        0.14        0.29        0.43        0.71        0.29        0.71   
         45        1.29        1.29        1.29        1.00        1.00        0.57   
         
             2017-11-19  2017-11-26  2017-12-03  
         37        3.00        2.57        0.00  
         39        0.00        0.00        0.00  
         40        0.00        0.00        0.00  
         44        0.00        0.00        0.00  
         45        0.00        0.00        0.00  
         
         [5 rows x 100 columns]
\end{Verbatim}
            
    \subsubsection{Rolling Average}\label{rolling-average}

Apply a rolling average filter using a window of size = 4 , then remove
rare sales

    \begin{Verbatim}[commandchars=\\\{\}]
{\color{incolor}In [{\color{incolor}28}]:} \PY{n}{window} \PY{o}{=} \PY{l+m+mi}{4}
\end{Verbatim}


    \begin{center}
    \adjustimage{max size={0.9\linewidth}{0.9\paperheight}}{output_13_0.png}
    \end{center}
    { \hspace*{\fill} \\}
    
    \begin{Verbatim}[commandchars=\\\{\}]
(2847, 97)

    \end{Verbatim}

    \subsubsection{Remove Outliers -
Winsorizing}\label{remove-outliers---winsorizing}

Winsorizing is to set all outliers to a specified percentile of the
data; for example, a 90\% winsorization would see all data below the 5th
percentile set to the 5th percentile, and data above the 95th percentile
set to the 95th percentile

    \begin{Verbatim}[commandchars=\\\{\}]
{\color{incolor}In [{\color{incolor}29}]:} \PY{n}{t} \PY{o}{=} \PY{l+m+mi}{4}
\end{Verbatim}


    \begin{Verbatim}[commandchars=\\\{\}]
C:\textbackslash{}Users\textbackslash{}rahmim00\textbackslash{}AppData\textbackslash{}Local\textbackslash{}Continuum\textbackslash{}anaconda3\textbackslash{}envs\textbackslash{}dev\_py34\textbackslash{}lib\textbackslash{}site-packages\textbackslash{}scipy\textbackslash{}stats\textbackslash{}stats.py:2245: RuntimeWarning: invalid value encountered in true\_divide
  np.expand\_dims(sstd, axis=axis))
C:\textbackslash{}Users\textbackslash{}rahmim00\textbackslash{}AppData\textbackslash{}Local\textbackslash{}Continuum\textbackslash{}anaconda3\textbackslash{}envs\textbackslash{}dev\_py34\textbackslash{}lib\textbackslash{}site-packages\textbackslash{}ipykernel\_launcher.py:6: RuntimeWarning: invalid value encountered in greater
  

    \end{Verbatim}

    \begin{center}
    \adjustimage{max size={0.9\linewidth}{0.9\paperheight}}{output_15_1.png}
    \end{center}
    { \hspace*{\fill} \\}
    
    \subsubsection{Mainly Zeros: Rare sales}\label{mainly-zeros-rare-sales}

Remove the products that werent sold for \textbf{at least 5 weeks}

    \begin{Verbatim}[commandchars=\\\{\}]
{\color{incolor}In [{\color{incolor}30}]:} \PY{n}{t} \PY{o}{=} \PY{l+m+mi}{5}
\end{Verbatim}


    \begin{center}
    \adjustimage{max size={0.9\linewidth}{0.9\paperheight}}{output_17_0.png}
    \end{center}
    { \hspace*{\fill} \\}
    
    \begin{Verbatim}[commandchars=\\\{\}]
Series With less than 5 values are removed
Removed: 1 , Remaining: 2846

    \end{Verbatim}

    \begin{Verbatim}[commandchars=\\\{\}]
{\color{incolor}In [{\color{incolor}31}]:} \PY{n}{product\PYZus{}df\PYZus{}keep} \PY{o}{=} \PY{n}{product\PYZus{}df}
         \PY{n+nb}{print}\PY{p}{(}\PY{n}{product\PYZus{}df}\PY{o}{.}\PY{n}{shape}\PY{p}{)}
         \PY{n}{product\PYZus{}df}\PY{o}{.}\PY{n}{head}\PY{p}{(}\PY{p}{)}
\end{Verbatim}


    \begin{Verbatim}[commandchars=\\\{\}]
(2846, 97)

    \end{Verbatim}

\begin{Verbatim}[commandchars=\\\{\}]
{\color{outcolor}Out[{\color{outcolor}31}]:}     2016-01-24  2016-01-31  2016-02-07  2016-02-14  2016-02-21  2016-02-28  \textbackslash{}
         37        4.46        4.46        4.46        4.46        4.46        4.46   
         39       10.11        9.54        7.64        6.25        5.96        5.71   
         40        1.29        0.43        0.43        0.43        0.50        0.93   
         44        1.18        1.61        1.61        1.79        1.89        1.64   
         45        0.96        0.96        1.07        1.21        1.14        1.14   
         
             2016-03-06  2016-03-13  2016-03-20  2016-03-27     {\ldots}      2017-09-24  \textbackslash{}
         37        4.46        4.46        4.25        4.32     {\ldots}            2.25   
         39        5.71        7.21       10.75       13.36     {\ldots}           11.68   
         40        1.89        3.04        3.07        2.79     {\ldots}            1.00   
         44        1.75        1.82        1.75        1.82     {\ldots}            0.46   
         45        1.07        0.96        0.93        0.93     {\ldots}            1.11   
         
             2017-10-01  2017-10-08  2017-10-15  2017-10-22  2017-10-29  2017-11-05  \textbackslash{}
         37        2.04        2.04        2.04        2.04        2.11        2.04   
         39        8.89       11.43       13.54       13.64       13.39       10.36   
         40        1.11        1.21        1.29        1.21        1.21        1.14   
         44        0.43        0.43        0.43        0.43        0.43        0.54   
         45        1.21        1.14        1.18        1.21        1.14        0.96   
         
             2017-11-12  2017-11-19  2017-11-26  
         37        2.29        2.36        2.04  
         39        6.04        5.29        5.29  
         40        0.82        0.57        0.43  
         44        0.43        0.43        0.43  
         45        0.71        0.71        0.71  
         
         [5 rows x 97 columns]
\end{Verbatim}
            
    \subsubsection{Normalize Data}\label{normalize-data}

We end up with 2846 product/client . Apply a Z-normalization (subtract
the mean and divide by the standard deviation) in order to get a
\textbf{scale-independent} data

    \begin{Verbatim}[commandchars=\\\{\}]
{\color{incolor}In [{\color{incolor}33}]:} \PY{n}{product\PYZus{}df\PYZus{}full}\PY{o}{.}\PY{n}{head}\PY{p}{(}\PY{p}{)}
\end{Verbatim}


\begin{Verbatim}[commandchars=\\\{\}]
{\color{outcolor}Out[{\color{outcolor}33}]:}            Product  Client  2016-01-24 00:00:00  2016-01-31 00:00:00  \textbackslash{}
         37  GBA001BND060SS  68A139                 1.61                 1.61   
         39  GBA001BND060SS  68C120                 0.51                 0.30   
         40  GBA001BND060SS  68C122                -0.15                -1.04   
         44  GBA001BND060SS  68C126                -0.11                 0.65   
         45  GBA001BND060SS  68C129                -0.35                -0.35   
         
             2016-02-07 00:00:00  2016-02-14 00:00:00  2016-02-21 00:00:00  \textbackslash{}
         37                 1.61                 1.61                 1.61   
         39                -0.40                -0.91                -1.01   
         40                -1.04                -1.04                -0.96   
         44                 0.65                 0.97                 1.16   
         45                 0.33                 1.25                 0.79   
         
             2016-02-28 00:00:00  2016-03-06 00:00:00  2016-03-13 00:00:00  \textbackslash{}
         37                 1.61                 1.61                 1.61   
         39                -1.10                -1.10                -0.55   
         40                -0.52                 0.48                 1.66   
         44                 0.71                 0.90                 1.03   
         45                 0.79                 0.33                -0.35   
         
                    {\ldots}           2017-09-24 00:00:00  2017-10-01 00:00:00  \textbackslash{}
         37         {\ldots}                         -1.00                -1.26   
         39         {\ldots}                          1.08                 0.06   
         40         {\ldots}                         -0.45                -0.33   
         44         {\ldots}                         -1.37                -1.43   
         45         {\ldots}                          0.56                 1.25   
         
             2017-10-08 00:00:00  2017-10-15 00:00:00  2017-10-22 00:00:00  \textbackslash{}
         37                -1.26                -1.26                -1.26   
         39                 0.99                 1.76                 1.80   
         40                -0.22                -0.15                -0.22   
         44                -1.43                -1.43                -1.43   
         45                 0.79                 1.02                 1.25   
         
             2017-10-29 00:00:00  2017-11-05 00:00:00  2017-11-12 00:00:00  \textbackslash{}
         37                -1.17                -1.26                -0.96   
         39                 1.71                 0.60                -0.98   
         40                -0.22                -0.30                -0.63   
         44                -1.43                -1.24                -1.43   
         45                 0.79                -0.35                -1.95   
         
             2017-11-19 00:00:00  2017-11-26 00:00:00  
         37                -0.88                -1.26  
         39                -1.26                -1.26  
         40                -0.89                -1.04  
         44                -1.43                -1.43  
         45                -1.95                -1.95  
         
         [5 rows x 99 columns]
\end{Verbatim}
            
    

    

    

    

    \section{2. Principal Components
Analysis}\label{principal-components-analysis}

L'Analyse en Composantes Principales est une méthode de projection qui
permet de projeter les observations dans un espace de dimensions
réduites tel qu'un maximum d'information soit conservée.

Si l'information associée aux 2 ou 3 premiers axes représente un
pourcentage suffisant de la variabilité totale du nuage de points, on
pourra représenter les observations sur un graphique à 2 ou 3
dimensions, facilitant ainsi grandement l'interprétation.

    \begin{Verbatim}[commandchars=\\\{\}]
{\color{incolor}In [{\color{incolor}34}]:} \PY{n}{hlp}\PY{o}{.}\PY{n}{Clusters\PYZus{}plot}\PY{p}{(}\PY{n}{X}\PY{o}{=} \PY{n}{X\PYZus{}pca}\PY{p}{,} \PY{n}{labels} \PY{o}{=} \PY{n}{np}\PY{o}{.}\PY{n}{zeros}\PY{p}{(}\PY{n+nb}{len}\PY{p}{(}\PY{n}{X\PYZus{}pca}\PY{p}{)}\PY{p}{)}\PY{p}{)}
\end{Verbatim}


    \begin{center}
    \adjustimage{max size={0.9\linewidth}{0.9\paperheight}}{output_26_0.png}
    \end{center}
    { \hspace*{\fill} \\}
    
    \begin{Verbatim}[commandchars=\\\{\}]
[ 0.192122    0.13360553  0.0868289   0.07429497  0.04438952]

    \end{Verbatim}

    \subsection{PCA 01}\label{pca-01}

Affiche une tendance croissante dans le temps.

    \begin{Verbatim}[commandchars=\\\{\}]
{\color{incolor}In [{\color{incolor}36}]:} \PY{n}{plt}\PY{o}{.}\PY{n}{plot}\PY{p}{(}\PY{n+nb}{range}\PY{p}{(}\PY{n}{nb\PYZus{}col}\PY{p}{)}\PY{p}{,} \PY{n}{princ\PYZus{}axis} \PY{p}{[}\PY{l+m+mi}{1}\PY{p}{,}\PY{p}{:}\PY{p}{]}\PY{p}{)}
\end{Verbatim}


    \begin{center}
    \adjustimage{max size={0.9\linewidth}{0.9\paperheight}}{output_28_0.png}
    \end{center}
    { \hspace*{\fill} \\}
    
    \subsection{PCA 02}\label{pca-02}

Oppose les ventes en terme de saisonnalité. Les produits corrélés avec
cet axe ont tendance à être vendu durant les périodes d'Automne ou début
d'été.

    \begin{Verbatim}[commandchars=\\\{\}]
{\color{incolor}In [{\color{incolor}37}]:} \PY{n}{plt}\PY{o}{.}\PY{n}{plot}\PY{p}{(}\PY{n+nb}{range}\PY{p}{(}\PY{n}{nb\PYZus{}col}\PY{p}{)}\PY{p}{,} \PY{n}{princ\PYZus{}axis} \PY{p}{[}\PY{l+m+mi}{2}\PY{p}{,}\PY{p}{:}\PY{p}{]}\PY{p}{)}
\end{Verbatim}


    \begin{center}
    \adjustimage{max size={0.9\linewidth}{0.9\paperheight}}{output_30_0.png}
    \end{center}
    { \hspace*{\fill} \\}
    
    \begin{Verbatim}[commandchars=\\\{\}]
{\color{incolor}In [{\color{incolor}49}]:} \PY{n}{corrSamples} \PY{o}{=} \PY{n}{hlp}\PY{o}{.}\PY{n}{GetMostCorrelatedTo}\PY{p}{(}\PY{n}{X\PYZus{}pca}\PY{p}{,}\PY{n}{component}\PY{p}{,}\PY{n}{index}\PY{o}{=}\PY{n}{product\PYZus{}df}\PY{o}{.}\PY{n}{index}\PY{p}{)}
\end{Verbatim}


    \begin{Verbatim}[commandchars=\\\{\}]
Produit 1744

    \end{Verbatim}

    \begin{center}
    \adjustimage{max size={0.9\linewidth}{0.9\paperheight}}{output_31_1.png}
    \end{center}
    { \hspace*{\fill} \\}
    
    

    

    

    \section{3. Modeling - Clustering
Algorithms}\label{modeling---clustering-algorithms}

Try out Hierarchical clustering, kMeans and kMedodis on raw (cleaned)
data. Then, do a \textbf{PCA plot} to visualize the result of the
clustering on the principal components

    \begin{Verbatim}[commandchars=\\\{\}]
{\color{incolor}In [{\color{incolor}61}]:} \PY{n}{product\PYZus{}df\PYZus{}full}\PY{o}{.}\PY{n}{head}\PY{p}{(}\PY{p}{)}
\end{Verbatim}


\begin{Verbatim}[commandchars=\\\{\}]
{\color{outcolor}Out[{\color{outcolor}61}]:}           Product  Client  2016-01-24 00:00:00  2016-01-31 00:00:00  \textbackslash{}
         0  GBA001BND060SS  68A139                 1.61                 1.61   
         1  GBA001BND060SS  68C120                 0.51                 0.30   
         2  GBA001BND060SS  68C122                -0.15                -1.04   
         3  GBA001BND060SS  68C126                -0.11                 0.65   
         4  GBA001BND060SS  68C129                -0.35                -0.35   
         
            2016-02-07 00:00:00  2016-02-14 00:00:00  2016-02-21 00:00:00  \textbackslash{}
         0                 1.61                 1.61                 1.61   
         1                -0.40                -0.91                -1.01   
         2                -1.04                -1.04                -0.96   
         3                 0.65                 0.97                 1.16   
         4                 0.33                 1.25                 0.79   
         
            2016-02-28 00:00:00  2016-03-06 00:00:00  2016-03-13 00:00:00  \textbackslash{}
         0                 1.61                 1.61                 1.61   
         1                -1.10                -1.10                -0.55   
         2                -0.52                 0.48                 1.66   
         3                 0.71                 0.90                 1.03   
         4                 0.79                 0.33                -0.35   
         
                   {\ldots}           2017-09-24 00:00:00  2017-10-01 00:00:00  \textbackslash{}
         0         {\ldots}                         -1.00                -1.26   
         1         {\ldots}                          1.08                 0.06   
         2         {\ldots}                         -0.45                -0.33   
         3         {\ldots}                         -1.37                -1.43   
         4         {\ldots}                          0.56                 1.25   
         
            2017-10-08 00:00:00  2017-10-15 00:00:00  2017-10-22 00:00:00  \textbackslash{}
         0                -1.26                -1.26                -1.26   
         1                 0.99                 1.76                 1.80   
         2                -0.22                -0.15                -0.22   
         3                -1.43                -1.43                -1.43   
         4                 0.79                 1.02                 1.25   
         
            2017-10-29 00:00:00  2017-11-05 00:00:00  2017-11-12 00:00:00  \textbackslash{}
         0                -1.17                -1.26                -0.96   
         1                 1.71                 0.60                -0.98   
         2                -0.22                -0.30                -0.63   
         3                -1.43                -1.24                -1.43   
         4                 0.79                -0.35                -1.95   
         
            2017-11-19 00:00:00  2017-11-26 00:00:00  
         0                -0.88                -1.26  
         1                -1.26                -1.26  
         2                -0.89                -1.04  
         3                -1.43                -1.43  
         4                -1.95                -1.95  
         
         [5 rows x 99 columns]
\end{Verbatim}
            
    \subsubsection{Métriques
d'évaluation}\label{muxe9triques-duxe9valuation}

\textbf{SSE:} La somme des carrés est une mesure de variation ou d'écart
par rapport à la moyenne. Elle représente la somme des carrés des
différences par rapport au centre du cluster.

\textbf{Silhouette Score}: Permet d'évaluer si chaque point appartient
au « bon » cluster : est-il proche des points du cluster auquel il
appartient ? Est-il loin des autres points ?

    \begin{Verbatim}[commandchars=\\\{\}]
{\color{incolor}In [{\color{incolor} }]:} \PY{k}{def} \PY{n+nf}{getSSE}\PY{p}{(}\PY{n}{samples}\PY{p}{,}\PY{n}{labels}\PY{p}{)}\PY{p}{:}
            \PY{k}{return} \PY{n}{np}\PY{o}{.}\PY{n}{sum}\PY{p}{(} \PY{p}{(}\PY{n}{samples}\PY{o}{\PYZhy{}}\PY{n}{labels}\PY{p}{)}\PY{o}{*}\PY{o}{*}\PY{l+m+mi}{2}\PY{p}{)}
        
        \PY{k}{def} \PY{n+nf}{getSilouhaite}\PY{p}{(}\PY{n}{samples}\PY{p}{,}\PY{n}{labels}\PY{p}{)}\PY{p}{:}
            \PY{k}{return} \PY{n}{metrics}\PY{o}{.}\PY{n}{silhouette\PYZus{}score}\PY{p}{(}\PY{n}{samples}\PY{p}{,}\PY{n}{labels}\PY{p}{)}
\end{Verbatim}


    \subsection{Agglomerative - (CAH)}\label{agglomerative---cah}

Le principe est de rassembler les individus qui affichent des courbes
similaires (en distance euclidienne ou en corrélation). Les distances
sont calculées entre les produits deux à deux. Plus deux observations
seront dissemblables, plus la distance sera importante. La CAH va
ensuite rassembler les individus de manière itérative afin de produire
un dendrogramme ou arbre de classification

    \begin{Verbatim}[commandchars=\\\{\}]
{\color{incolor}In [{\color{incolor}59}]:} \PY{n}{hierarchy}\PY{o}{.}\PY{n}{linkage}\PY{p}{(}\PY{n}{X\PYZus{}z}\PY{p}{,} \PY{n}{method}\PY{o}{=}\PY{l+s+s1}{\PYZsq{}}\PY{l+s+s1}{complete}\PY{l+s+s1}{\PYZsq{}}\PY{p}{,}\PY{n}{metric}\PY{o}{=}\PY{l+s+s1}{\PYZsq{}}\PY{l+s+s1}{euclidean}\PY{l+s+s1}{\PYZsq{}}\PY{p}{)}
         \PY{n}{hierarchy}\PY{o}{.}\PY{n}{dendrogram}\PY{p}{(}\PY{n}{Z}\PY{p}{,} \PY{n}{truncate\PYZus{}mode}\PY{o}{=}\PY{l+s+s1}{\PYZsq{}}\PY{l+s+s1}{lastp}\PY{l+s+s1}{\PYZsq{}}\PY{p}{,} \PY{n}{p}\PY{o}{=}\PY{l+m+mi}{100}\PY{p}{,} \PY{n}{leaf\PYZus{}rotation}\PY{o}{=}\PY{l+m+mf}{90.}\PY{p}{,} \PY{n}{leaf\PYZus{}font\PYZus{}size}\PY{o}{=}\PY{l+m+mf}{7.}\PY{p}{,} \PY{n}{show\PYZus{}contracted}\PY{o}{=}\PY{k+kc}{True}\PY{p}{)}
\end{Verbatim}


    \begin{center}
    \adjustimage{max size={0.9\linewidth}{0.9\paperheight}}{output_40_0.png}
    \end{center}
    { \hspace*{\fill} \\}
    
    \subsubsection{Critère de Ward}\label{crituxe8re-de-ward}

La méthode de Ward consiste à regrouper les classes de façon que
l'augmentation de l'inertie interclasse soit maximum.

    \begin{Verbatim}[commandchars=\\\{\}]
{\color{incolor}In [{\color{incolor}72}]:} \PY{n}{n\PYZus{}cluster} \PY{o}{=} \PY{l+m+mi}{100}
         \PY{n}{ward} \PY{o}{=} \PY{n}{AgglomerativeClustering}\PY{p}{(}\PY{n}{n\PYZus{}clusters}\PY{o}{=}\PY{n}{n\PYZus{}cluster}\PY{p}{,} \PY{n}{linkage}\PY{o}{=}\PY{l+s+s1}{\PYZsq{}}\PY{l+s+s1}{ward}\PY{l+s+s1}{\PYZsq{}}\PY{p}{)}\PY{o}{.}\PY{n}{fit}\PY{p}{(}\PY{n}{X\PYZus{}z}\PY{p}{)}
\end{Verbatim}


    \begin{center}
    \adjustimage{max size={0.9\linewidth}{0.9\paperheight}}{output_42_0.png}
    \end{center}
    { \hspace*{\fill} \\}
    
    \subsection{BIRCH Algorithm}\label{birch-algorithm}

Used to perform hierarchical clustering over particularly large
data-sets.The advantage of using BIRCH algorithm is its ability to
incrementally \& dynamically cluster incoming.

    \begin{Verbatim}[commandchars=\\\{\}]
{\color{incolor}In [{\color{incolor}75}]:} \PY{n}{labels\PYZus{}birch} \PY{o}{=} \PY{n}{Birch}\PY{p}{(}\PY{n}{n\PYZus{}clusters}\PY{o}{=} \PY{n}{n\PYZus{}cluster}\PY{p}{,} \PY{n}{threshold}\PY{o}{=}\PY{l+m+mf}{0.5}\PY{p}{,} \PY{n}{compute\PYZus{}labels}\PY{o}{=}\PY{k+kc}{True}\PY{p}{)}\PY{o}{.}\PY{n}{fit\PYZus{}predict}\PY{p}{(}\PY{n}{X\PYZus{}z}\PY{p}{)}
\end{Verbatim}


    \begin{center}
    \adjustimage{max size={0.9\linewidth}{0.9\paperheight}}{output_44_0.png}
    \end{center}
    { \hspace*{\fill} \\}
    
    \subsection{Partitionning Algorithms}\label{partitionning-algorithms}

    \subsubsection{SOM (Cartes topologiques
auto-organisatrices)}\label{som-cartes-topologiques-auto-organisatrices}

Les cartes auto-organisatrices sont constituées d'une grille. Dans
chaque nœud de la grille se trouve un « neurone ». Chaque neurone est
lié à un vecteur référent, responsable d'une zone dans l'espace des
données (Un certain nombre de courbes similaires).

    \begin{Verbatim}[commandchars=\\\{\}]
{\color{incolor}In [{\color{incolor}67}]:} \PY{n}{som} \PY{o}{=} \PY{n}{sompy}\PY{o}{.}\PY{n}{SOMFactory}\PY{p}{(}\PY{p}{)}\PY{o}{.}\PY{n}{build}\PY{p}{(}\PY{n}{X\PYZus{}z}\PY{p}{,} \PY{n}{mapsize}\PY{p}{,} \PY{n}{mask}\PY{o}{=}\PY{k+kc}{None}\PY{p}{,} \PY{n}{mapshape}\PY{o}{=}\PY{l+s+s1}{\PYZsq{}}\PY{l+s+s1}{rectangular}\PY{l+s+s1}{\PYZsq{}}\PY{p}{,} \PY{n}{lattice}\PY{o}{=}\PY{l+s+s1}{\PYZsq{}}\PY{l+s+s1}{rect}\PY{l+s+s1}{\PYZsq{}}\PY{p}{,} \PY{n}{normalization}\PY{o}{=}\PY{l+s+s1}{\PYZsq{}}\PY{l+s+s1}{var}\PY{l+s+s1}{\PYZsq{}}\PY{p}{,} \PY{n}{initialization}\PY{o}{=}\PY{l+s+s1}{\PYZsq{}}\PY{l+s+s1}{pca}\PY{l+s+s1}{\PYZsq{}}\PY{p}{,} \PY{n}{neighborhood}\PY{o}{=}\PY{l+s+s1}{\PYZsq{}}\PY{l+s+s1}{gaussian}\PY{l+s+s1}{\PYZsq{}}\PY{p}{,} \PY{n}{training}\PY{o}{=}\PY{l+s+s1}{\PYZsq{}}\PY{l+s+s1}{batch}\PY{l+s+s1}{\PYZsq{}}\PY{p}{,} \PY{n}{name}\PY{o}{=}\PY{l+s+s1}{\PYZsq{}}\PY{l+s+s1}{sompy}\PY{l+s+s1}{\PYZsq{}}\PY{p}{)}  
\end{Verbatim}


    \begin{Verbatim}[commandchars=\\\{\}]
maxtrainlen \%d inf
maxtrainlen \%d inf
Topographic error = 0.153900210822; Quantization error = 6.2435798326

    \end{Verbatim}

    \begin{center}
    \adjustimage{max size={0.9\linewidth}{0.9\paperheight}}{output_47_1.png}
    \end{center}
    { \hspace*{\fill} \\}
    
    \subsection{K-Means Algorithm}\label{k-means-algorithm}

    The main idea is to define k centers, one for each cluster. then take
each point belonging to a given data set and associate it to the nearest
center. When no point is pending, re-calculate k new centroids as
barycenter of the clusters. A new binding has to be done between the
same data set points and the nearest new center. repeat until no more
changes are done ie centers do not move any more.

    \subsubsection{K-means: Validate the number of
clusters}\label{k-means-validate-the-number-of-clusters}

    \begin{Verbatim}[commandchars=\\\{\}]
{\color{incolor}In [{\color{incolor}81}]:} \PY{n}{plot\PYZus{}inertia}\PY{p}{(}\PY{p}{)}
         \PY{n}{plot\PYZus{}silhouaite}\PY{p}{(}\PY{p}{)}
\end{Verbatim}


    \begin{center}
    \adjustimage{max size={0.9\linewidth}{0.9\paperheight}}{output_51_0.png}
    \end{center}
    { \hspace*{\fill} \\}
    
    \subsubsection{Apply K-means model}\label{apply-k-means-model}

    \begin{Verbatim}[commandchars=\\\{\}]
{\color{incolor}In [{\color{incolor}101}]:} \PY{n}{kmeans} \PY{o}{=} \PY{n}{KMeans}\PY{p}{(}\PY{n}{n\PYZus{}clusters}\PY{o}{=}\PY{l+m+mi}{81}\PY{p}{)}\PY{o}{.}\PY{n}{fit}\PY{p}{(}\PY{n}{X\PYZus{}z}\PY{p}{)}
\end{Verbatim}


    \begin{center}
    \adjustimage{max size={0.9\linewidth}{0.9\paperheight}}{output_53_0.png}
    \end{center}
    { \hspace*{\fill} \\}
    
    \subsection{K-medoids (PAM Algorithm)}\label{k-medoids-pam-algorithm}

The k-medoids algorithm is a clustering algorithm related to the k-means
algorithm. K-means attempts to minimize the total squared error, while
k-medoids minimizes the sum of dissimilarities between points labeled to
be in a cluster and a point designated as the center of that cluster. In
contrast to the k-means algorithm, k-medoids chooses datapoints as
centers ( medoids or exemplars).

    \subsubsection{K-medoids Validate the number of
clusters}\label{k-medoids-validate-the-number-of-clusters}

    \paragraph{Custom distances
calculation}\label{custom-distances-calculation}

    \begin{Verbatim}[commandchars=\\\{\}]
{\color{incolor}In [{\color{incolor}103}]:} \PY{n}{corr\PYZus{}distance} \PY{o}{=} \PY{n}{squareform}\PY{p}{(}\PY{n}{pdist}\PY{p}{(}\PY{n}{X\PYZus{}z}\PY{p}{,} \PY{l+s+s1}{\PYZsq{}}\PY{l+s+s1}{correlation}\PY{l+s+s1}{\PYZsq{}}\PY{p}{)}\PY{p}{)} \PY{c+c1}{\PYZsh{}Pearsons}
          \PY{n}{euclid\PYZus{}distance} \PY{o}{=} \PY{n}{squareform}\PY{p}{(}\PY{n}{pdist}\PY{p}{(}\PY{n}{X\PYZus{}z}\PY{p}{,} \PY{l+s+s1}{\PYZsq{}}\PY{l+s+s1}{euclidean}\PY{l+s+s1}{\PYZsq{}}\PY{p}{)}\PY{p}{)}
\end{Verbatim}


    \begin{Verbatim}[commandchars=\\\{\}]
{\color{incolor}In [{\color{incolor}139}]:} \PY{c+c1}{\PYZsh{} plot\PYZus{}inertia()}
          \PY{c+c1}{\PYZsh{} plot\PYZus{}silhouaite()}
          
          \PY{o}{\PYZpc{}}\PY{k}{matplotlib} inline
          \PY{n}{clusters}\PY{o}{=} \PY{n}{np}\PY{o}{.}\PY{n}{linspace}\PY{p}{(}\PY{l+m+mi}{91}\PY{p}{,}\PY{l+m+mi}{120}\PY{p}{,}\PY{l+m+mi}{15}\PY{p}{)}\PY{o}{.}\PY{n}{astype}\PY{p}{(}\PY{n+nb}{int}\PY{p}{)}
          \PY{n}{silouhaite} \PY{o}{=} \PY{p}{[}\PY{p}{]}
          \PY{n}{inertia} \PY{o}{=} \PY{p}{[}\PY{p}{]}
          \PY{k}{for} \PY{n}{cluster} \PY{o+ow}{in} \PY{n}{clusters}\PY{p}{:}
              \PY{n}{labels}\PY{p}{,} \PY{n}{medoids} \PY{o}{=} \PY{n}{kMedoids}\PY{o}{.}\PY{n}{cluster}\PY{p}{(}\PY{n}{euclid\PYZus{}distance}\PY{p}{,}\PY{n}{k}\PY{o}{=} \PY{n}{cluster}\PY{p}{)}
              \PY{n}{silouhaite}\PY{o}{.}\PY{n}{append}\PY{p}{(}\PY{n}{hlp}\PY{o}{.}\PY{n}{getSilouhaite}\PY{p}{(}\PY{n}{X\PYZus{}z}\PY{p}{,}\PY{n}{labels}\PY{p}{)}\PY{p}{)}
              \PY{n}{sse} \PY{o}{=} \PY{n}{hlp}\PY{o}{.}\PY{n}{getSSE}\PY{p}{(}\PY{n}{X\PYZus{}z}\PY{p}{,}\PY{n}{X\PYZus{}z}\PY{p}{[}\PY{n}{labels}\PY{p}{]}\PY{p}{)}
              \PY{n}{inertia}\PY{o}{.}\PY{n}{append}\PY{p}{(}\PY{n}{np}\PY{o}{.}\PY{n}{sqrt}\PY{p}{(}\PY{n}{sse}\PY{o}{/}\PY{n+nb}{len}\PY{p}{(}\PY{n}{labels}\PY{p}{)}\PY{p}{)}\PY{p}{)}
              
          
          \PY{n}{plt}\PY{o}{.}\PY{n}{figure}\PY{p}{(}\PY{n}{figsize}\PY{o}{=}\PY{p}{(}\PY{l+m+mi}{16}\PY{p}{,}\PY{l+m+mi}{4}\PY{p}{)}\PY{p}{)}
              
          \PY{n}{plt}\PY{o}{.}\PY{n}{subplot}\PY{p}{(}\PY{l+m+mi}{1}\PY{p}{,}\PY{l+m+mi}{2}\PY{p}{,}\PY{l+m+mi}{1}\PY{p}{)}
          \PY{n}{inertia} \PY{o}{=} \PY{n}{np}\PY{o}{.}\PY{n}{array}\PY{p}{(}\PY{n}{inertia}\PY{p}{)}
          \PY{n}{plt}\PY{o}{.}\PY{n}{title}\PY{p}{(}\PY{l+s+s2}{\PYZdq{}}\PY{l+s+s2}{Total inertia according to clusters}\PY{l+s+s2}{\PYZdq{}}\PY{p}{)}    
          \PY{n}{plt}\PY{o}{.}\PY{n}{plot}\PY{p}{(}\PY{n}{np}\PY{o}{.}\PY{n}{arange}\PY{p}{(}\PY{l+m+mi}{0}\PY{p}{,}\PY{n+nb}{len}\PY{p}{(}\PY{n}{clusters}\PY{p}{)}\PY{p}{)}\PY{p}{,}\PY{n}{inertia}\PY{p}{)}\PY{c+c1}{\PYZsh{}scale it to acc2}
          \PY{n}{plt}\PY{o}{.}\PY{n}{xticks}\PY{p}{(}\PY{n}{np}\PY{o}{.}\PY{n}{arange}\PY{p}{(}\PY{l+m+mi}{0}\PY{p}{,}\PY{n+nb}{len}\PY{p}{(}\PY{n}{clusters}\PY{p}{)}\PY{p}{)}\PY{p}{,}\PY{n}{clusters}\PY{p}{)}
          
          
          \PY{n}{acc} \PY{o}{=} \PY{n}{np}\PY{o}{.}\PY{n}{diff}\PY{p}{(}\PY{n}{inertia}\PY{p}{,} \PY{l+m+mi}{2}\PY{p}{)}  \PY{c+c1}{\PYZsh{} 2nd derivative of the inertia curve}
          \PY{c+c1}{\PYZsh{}plt.plot(np.arange(2,len(clusters)), acc)}
          \PY{n}{best\PYZus{}ks} \PY{o}{=} \PY{n}{acc}\PY{o}{.}\PY{n}{argsort}\PY{p}{(}\PY{p}{)}\PY{p}{[}\PY{p}{:}\PY{p}{:}\PY{o}{\PYZhy{}}\PY{l+m+mi}{1}\PY{p}{]}
          \PY{n}{k} \PY{o}{=}  \PY{n}{best\PYZus{}ks}\PY{o}{+} \PY{l+m+mi}{2}  \PY{c+c1}{\PYZsh{} if idx 0 is the max of this we want 2 clusters}
          \PY{c+c1}{\PYZsh{} print (\PYZdq{}clusters:\PYZdq{},clusters[k])}
          
          
          
          \PY{n}{plt}\PY{o}{.}\PY{n}{subplot}\PY{p}{(}\PY{l+m+mi}{1}\PY{p}{,}\PY{l+m+mi}{2}\PY{p}{,}\PY{l+m+mi}{2}\PY{p}{)}
          \PY{n}{silouhaite} \PY{o}{=} \PY{n}{np}\PY{o}{.}\PY{n}{array}\PY{p}{(}\PY{n}{silouhaite}\PY{p}{)}
          \PY{n}{plt}\PY{o}{.}\PY{n}{title}\PY{p}{(}\PY{l+s+s2}{\PYZdq{}}\PY{l+s+s2}{Silouhaite score according to clusters}\PY{l+s+s2}{\PYZdq{}}\PY{p}{)}    
          \PY{n}{plt}\PY{o}{.}\PY{n}{plot}\PY{p}{(}\PY{n}{np}\PY{o}{.}\PY{n}{arange}\PY{p}{(}\PY{l+m+mi}{0}\PY{p}{,}\PY{n+nb}{len}\PY{p}{(}\PY{n}{clusters}\PY{p}{)}\PY{p}{)}\PY{p}{,}\PY{n}{silouhaite}\PY{p}{)}\PY{c+c1}{\PYZsh{}scale it to acc2}
          \PY{n}{plt}\PY{o}{.}\PY{n}{xticks}\PY{p}{(}\PY{n}{np}\PY{o}{.}\PY{n}{arange}\PY{p}{(}\PY{l+m+mi}{0}\PY{p}{,}\PY{n+nb}{len}\PY{p}{(}\PY{n}{clusters}\PY{p}{)}\PY{p}{)}\PY{p}{,}\PY{n}{clusters}\PY{p}{)}
          \PY{n}{best\PYZus{}ks} \PY{o}{=} \PY{n}{silouhaite}\PY{o}{.}\PY{n}{argsort}\PY{p}{(}\PY{p}{)}\PY{p}{[}\PY{p}{:}\PY{p}{:}\PY{o}{\PYZhy{}}\PY{l+m+mi}{1}\PY{p}{]}
          \PY{c+c1}{\PYZsh{} print(\PYZdq{}clusters:\PYZdq{},clusters[best\PYZus{}ks])}
\end{Verbatim}


    \begin{center}
    \adjustimage{max size={0.9\linewidth}{0.9\paperheight}}{output_58_0.png}
    \end{center}
    { \hspace*{\fill} \\}
    
    \begin{Verbatim}[commandchars=\\\{\}]
{\color{incolor}In [{\color{incolor}144}]:} \PY{n}{n\PYZus{}cluster} \PY{o}{=} \PY{l+m+mi}{113}
          \PY{n}{label}\PY{p}{,} \PY{n}{medoids\PYZus{}euc} \PY{o}{=} \PY{n}{kMedoids}\PY{o}{.}\PY{n}{cluster}\PY{p}{(}\PY{n}{euclid\PYZus{}distance}\PY{p}{,}\PY{n}{k}\PY{o}{=} \PY{n}{n\PYZus{}cluster}\PY{p}{)}
          \PY{n}{labels\PYZus{}kmedoids} \PY{o}{=} \PY{n}{label}
          
          \PY{n}{labels\PYZus{}kmedoids\PYZus{}corr}\PY{p}{,}\PY{n}{medoids\PYZus{}corr} \PY{o}{=} \PY{n}{kMedoids}\PY{o}{.}\PY{n}{cluster}\PY{p}{(}\PY{n}{corr\PYZus{}distance}\PY{p}{,}\PY{n}{k}\PY{o}{=} \PY{n}{n\PYZus{}cluster}\PY{p}{)}
          \PY{n}{labels\PYZus{}kmedoids\PYZus{}spear}\PY{p}{,}\PY{n}{medoids\PYZus{}spear} \PY{o}{=} \PY{n}{kMedoids}\PY{o}{.}\PY{n}{cluster}\PY{p}{(}\PY{n}{corr\PYZus{}distance}\PY{p}{,}\PY{n}{k}\PY{o}{=} \PY{n}{n\PYZus{}cluster}\PY{p}{)}
          
          
          \PY{n}{SSE}\PY{p}{[}\PY{l+s+s2}{\PYZdq{}}\PY{l+s+s2}{kMedoids}\PY{l+s+s2}{\PYZdq{}}\PY{p}{]} \PY{o}{=} \PY{n}{hlp}\PY{o}{.}\PY{n}{getSSE}\PY{p}{(}\PY{n}{X\PYZus{}z}\PY{p}{,}\PY{n}{X\PYZus{}z}\PY{p}{[}\PY{n}{labels\PYZus{}kmedoids}\PY{p}{]}\PY{p}{)}
          \PY{n}{SSE}\PY{p}{[}\PY{l+s+s2}{\PYZdq{}}\PY{l+s+s2}{kMedoids\PYZus{}corr}\PY{l+s+s2}{\PYZdq{}}\PY{p}{]} \PY{o}{=} \PY{n}{hlp}\PY{o}{.}\PY{n}{getSSE}\PY{p}{(}\PY{n}{X\PYZus{}z}\PY{p}{,}\PY{n}{X\PYZus{}z}\PY{p}{[}\PY{n}{labels\PYZus{}kmedoids\PYZus{}corr}\PY{p}{]}\PY{p}{)}
          \PY{n}{SSE}\PY{p}{[}\PY{l+s+s2}{\PYZdq{}}\PY{l+s+s2}{kMedoids\PYZus{}spear}\PY{l+s+s2}{\PYZdq{}}\PY{p}{]} \PY{o}{=} \PY{n}{hlp}\PY{o}{.}\PY{n}{getSSE}\PY{p}{(}\PY{n}{X\PYZus{}z}\PY{p}{,}\PY{n}{X\PYZus{}z}\PY{p}{[}\PY{n}{labels\PYZus{}kmedoids\PYZus{}spear}\PY{p}{]}\PY{p}{)}
          
          
          \PY{n}{SILOUHAITE}\PY{p}{[}\PY{l+s+s2}{\PYZdq{}}\PY{l+s+s2}{kMedoids}\PY{l+s+s2}{\PYZdq{}}\PY{p}{]} \PY{o}{=} \PY{n}{hlp}\PY{o}{.}\PY{n}{getSilouhaite}\PY{p}{(}\PY{n}{X\PYZus{}z}\PY{p}{,}\PY{n}{labels\PYZus{}kmedoids}\PY{p}{)}
          \PY{n}{SILOUHAITE}\PY{p}{[}\PY{l+s+s2}{\PYZdq{}}\PY{l+s+s2}{kMedoids\PYZus{}corr}\PY{l+s+s2}{\PYZdq{}}\PY{p}{]} \PY{o}{=} \PY{n}{hlp}\PY{o}{.}\PY{n}{getSilouhaite}\PY{p}{(}\PY{n}{X\PYZus{}z}\PY{p}{,}\PY{n}{labels\PYZus{}kmedoids\PYZus{}corr}\PY{p}{)}
          \PY{n}{SILOUHAITE}\PY{p}{[}\PY{l+s+s2}{\PYZdq{}}\PY{l+s+s2}{kMedoids\PYZus{}spear}\PY{l+s+s2}{\PYZdq{}}\PY{p}{]} \PY{o}{=} \PY{n}{hlp}\PY{o}{.}\PY{n}{getSilouhaite}\PY{p}{(}\PY{n}{X\PYZus{}z}\PY{p}{,}\PY{n}{labels\PYZus{}kmedoids\PYZus{}spear}\PY{p}{)}
          
          
          \PY{n}{hlp}\PY{o}{.}\PY{n}{Clusters\PYZus{}plot}\PY{p}{(}\PY{n}{X}\PY{o}{=} \PY{n}{X\PYZus{}pca}\PY{p}{,} \PY{n}{labels} \PY{o}{=} \PY{n}{label}
                            \PY{p}{,}\PY{n}{info}\PY{o}{=}\PY{p}{[}\PY{l+s+s2}{\PYZdq{}}\PY{l+s+s2}{PCA K\PYZhy{}Medoids}\PY{l+s+s2}{\PYZdq{}}\PY{p}{,}\PY{l+s+s2}{\PYZdq{}}\PY{l+s+s2}{Euclidienne}\PY{l+s+s2}{\PYZdq{}}\PY{p}{,}\PY{l+s+s2}{\PYZdq{}}\PY{l+s+si}{\PYZpc{}d}\PY{l+s+s2}{ clusters :inertia }\PY{l+s+si}{\PYZpc{}.2f}\PY{l+s+s2}{\PYZdq{}}\PY{o}{\PYZpc{}}\PY{p}{(}\PY{n+nb}{len}\PY{p}{(}\PY{n+nb}{set}\PY{p}{(}\PY{n}{label}\PY{p}{)}\PY{p}{)}\PY{p}{,}\PY{n}{SSE}\PY{p}{[}\PY{l+s+s2}{\PYZdq{}}\PY{l+s+s2}{kMedoids}\PY{l+s+s2}{\PYZdq{}}\PY{p}{]}\PY{p}{)}\PY{p}{]}\PY{p}{)}
\end{Verbatim}


    \begin{center}
    \adjustimage{max size={0.9\linewidth}{0.9\paperheight}}{output_59_0.png}
    \end{center}
    { \hspace*{\fill} \\}
    
    \begin{Verbatim}[commandchars=\\\{\}]
{\color{incolor}In [{\color{incolor}120}]:} \PY{n}{df}\PY{o}{.}\PY{n}{nlargest}\PY{p}{(}\PY{l+m+mi}{5}\PY{p}{,}\PY{l+s+s2}{\PYZdq{}}\PY{l+s+s2}{SSE}\PY{l+s+s2}{\PYZdq{}}\PY{p}{)}
\end{Verbatim}


\begin{Verbatim}[commandchars=\\\{\}]
{\color{outcolor}Out[{\color{outcolor}120}]:}                     SSE
          Agg\_complete 521,110.11
          Ward         510,172.34
          Birch        504,761.40
          kMeans       500,011.45
          kMedoids     169,842.52
\end{Verbatim}
            
    \section{Display Clustering results}\label{display-clustering-results}

    \begin{Verbatim}[commandchars=\\\{\}]
{\color{incolor}In [{\color{incolor}125}]:} \PY{n}{disp} \PY{o}{=} \PY{n}{hlp}\PY{o}{.}\PY{n}{Cluster\PYZus{}series\PYZus{}plot}\PY{p}{(}\PY{n}{data\PYZus{}df} \PY{o}{=} \PY{n}{product\PYZus{}df\PYZus{}full}\PY{p}{,} \PY{n}{cluster\PYZus{}df} \PY{o}{=} \PY{n}{eucl\PYZus{}df}\PY{p}{,}\PY{n}{headers} \PY{o}{=} \PY{n}{row\PYZus{}headers}\PY{p}{)}
\end{Verbatim}


    \begin{center}
    \adjustimage{max size={0.9\linewidth}{0.9\paperheight}}{output_62_0.png}
    \end{center}
    { \hspace*{\fill} \\}
    
    

    

    

    \section{Forecast quality on
clusters}\label{forecast-quality-on-clusters}

In this section we'll build forecasts on a \textbf{cluster level} using
classical Holt-Winter's models. Our goal is to evaluate the quality of
the prediction on individual series composing the each cluster after
performing a \textbf{weighted split}.

The evaluation metric used is the \textbf{MASE (mean absolute scaled
error)} which is a scale-independant metric calculated wrt the MAE

    \begin{Verbatim}[commandchars=\\\{\}]
{\color{incolor}In [{\color{incolor}133}]:} \PY{n}{group\PYZus{}series}\PY{o}{.}\PY{n}{head}\PY{p}{(}\PY{p}{)}
\end{Verbatim}


\begin{Verbatim}[commandchars=\\\{\}]
{\color{outcolor}Out[{\color{outcolor}133}]:}          2016-01-10 00:00:00  2016-01-17 00:00:00  2016-01-24 00:00:00  \textbackslash{}
          Cluster                                                                  
          1                        401                  467                  422   
          2                         97                  103                   66   
          3                        371                  276                  119   
          4                       1076                 1185                  491   
          5                        281                  286                  256   
          
                   2016-01-31 00:00:00  2016-02-07 00:00:00  2016-02-14 00:00:00  \textbackslash{}
          Cluster                                                                  
          1                        520                  466                  219   
          2                         36                   37                   36   
          3                        507                  515                  148   
          4                        278                  251                 1624   
          5                        298                  298                  338   
          
                   2016-02-21 00:00:00  2016-02-28 00:00:00  2016-03-06 00:00:00  \textbackslash{}
          Cluster                                                                  
          1                        253                  286                  283   
          2                         46                  296                  379   
          3                        128                  128                  101   
          4                       1366                  451                 1354   
          5                        290                  322                  324   
          
                   2016-03-13 00:00:00         {\ldots}           2017-10-01 00:00:00  \textbackslash{}
          Cluster                              {\ldots}                                 
          1                        506         {\ldots}                           350   
          2                        109         {\ldots}                            37   
          3                        523         {\ldots}                           498   
          4                       1260         {\ldots}                           173   
          5                        332         {\ldots}                           393   
          
                   2017-10-08 00:00:00  2017-10-15 00:00:00  2017-10-22 00:00:00  \textbackslash{}
          Cluster                                                                  
          1                        422                  224                  232   
          2                        130                   94                   27   
          3                        466                  476                  170   
          4                        139                  703                  678   
          5                        311                  328                  429   
          
                   2017-10-29 00:00:00  2017-11-05 00:00:00  2017-11-12 00:00:00  \textbackslash{}
          Cluster                                                                  
          1                        179                  193                  192   
          2                         15                   13                   10   
          3                        122                  116                  743   
          4                        219                  154                  161   
          5                        311                  288                  249   
          
                   2017-11-19 00:00:00  2017-11-26 00:00:00  2017-12-03 00:00:00  
          Cluster                                                                 
          1                        107                  134                   19  
          2                         10                   11                    2  
          3                        183                  168                   21  
          4                        125                  268                   84  
          5                        143                  211                   34  
          
          [5 rows x 100 columns]
\end{Verbatim}
            
    \subsection{Building Holt-Winter's
Models}\label{building-holt-winters-models}

    \begin{Verbatim}[commandchars=\\\{\}]
{\color{incolor}In [{\color{incolor}141}]:} \PY{n}{forecast} \PY{o}{=} \PY{n}{pd}\PY{o}{.}\PY{n}{DataFrame}\PY{p}{(}\PY{n}{forecasts}\PY{p}{,}\PY{n}{columns} \PY{o}{=} \PY{n}{group\PYZus{}series}\PY{o}{.}\PY{n}{columns}\PY{p}{[}\PY{o}{\PYZhy{}}\PY{n}{fc}\PY{p}{:}\PY{p}{]}\PY{p}{,}\PY{n}{index} \PY{o}{=} \PY{n}{group\PYZus{}series}\PY{o}{.}\PY{n}{index}\PY{p}{)}
\end{Verbatim}


    \begin{center}
    \adjustimage{max size={0.9\linewidth}{0.9\paperheight}}{output_69_0.png}
    \end{center}
    { \hspace*{\fill} \\}
    
    \subsection{Forecast Split Forecast}\label{forecast-split-forecast}

    \begin{Verbatim}[commandchars=\\\{\}]
{\color{incolor}In [{\color{incolor}152}]:} \PY{n}{display}\PY{p}{(}\PY{n}{mase\PYZus{}variation}\PY{o}{.}\PY{n}{describe}\PY{p}{(}\PY{p}{)}\PY{p}{)}
\end{Verbatim}


    
    \begin{verbatim}
         Agg  Split  Forecast  Increase   Bias
count 113.00 113.00    113.00    113.00 113.00
mean    1.28   1.24      1.87      7.92 -27.62
std     0.69   0.73      1.32     52.28  22.98
min     0.25   0.46      0.71    -61.35 -81.73
25%     0.88   0.84      1.14    -22.37 -39.27
50%     1.09   1.05      1.46     -5.56 -28.47
75%     1.56   1.35      2.06     23.26 -17.14
max     3.60   4.89      8.66    286.80  54.02
    \end{verbatim}

    
    \begin{center}
    \adjustimage{max size={0.9\linewidth}{0.9\paperheight}}{output_71_1.png}
    \end{center}
    { \hspace*{\fill} \\}
    
    \begin{center}
    \adjustimage{max size={0.9\linewidth}{0.9\paperheight}}{output_71_2.png}
    \end{center}
    { \hspace*{\fill} \\}
    
    \subsection{More numbers...}\label{more-numbers...}

\begin{enumerate}
\def\labelenumi{\arabic{enumi}.}
\item
  The percentage of clusters where the forecast quality increases at the
  split level: We compare the average forecast quality at the
  \textbf{split level} to the forecast on \textbf{the aggregation level}
  of the cluster.
\item
  The percentage of "good" forecasts where the quality decreases with
  less than 20\%
\item
  The percentage of products where the \textbf{split forecast} is better
  than a \textbf{specific forecast}
\end{enumerate}

    \begin{Verbatim}[commandchars=\\\{\}]
{\color{incolor}In [{\color{incolor}167}]:} \PY{n}{display}\PY{p}{(}\PY{n}{mase\PYZus{}variation}\PY{p}{[}\PY{n}{good\PYZus{}splits}\PY{p}{]}\PY{o}{.}\PY{n}{sort\PYZus{}values}\PY{p}{(}\PY{n}{by}\PY{o}{=}\PY{p}{[}\PY{l+s+s2}{\PYZdq{}}\PY{l+s+s2}{Agg}\PY{l+s+s2}{\PYZdq{}}\PY{p}{]}\PY{p}{)}\PY{o}{.}\PY{n}{head}\PY{p}{(}\PY{p}{)}\PY{p}{)}
\end{Verbatim}


    \begin{Verbatim}[commandchars=\\\{\}]
Increased Forecast quality: 55.75\%
Good Forecast quality: 77.88\%
Better than detail Forecast quality: 90.27\%

    \end{Verbatim}

    
    \begin{verbatim}
    Agg  Split  Forecast  Increase   Bias
18 0.25   0.53      1.00    112.72 -46.38
91 0.35   0.57      0.91     64.06 -37.26
99 0.42   0.46      0.71      9.42 -35.83
7  0.42   0.57      1.52     36.52 -62.34
68 0.48   0.74      1.11     54.20 -32.78
    \end{verbatim}

    
    \subsection{Displaying forecasts}\label{displaying-forecasts}

    \begin{Verbatim}[commandchars=\\\{\}]
{\color{incolor}In [{\color{incolor}163}]:} \PY{n}{cluster} \PY{o}{=} \PY{l+m+mi}{99}
          \PY{n}{plt}\PY{o}{.}\PY{n}{show}\PY{p}{(}\PY{p}{)}
\end{Verbatim}


    \begin{Verbatim}[commandchars=\\\{\}]
Cluster 99 split:0.46 , agg:0.42 *** Increase 9.42 \%

    \end{Verbatim}

    \begin{center}
    \adjustimage{max size={0.9\linewidth}{0.9\paperheight}}{output_75_1.png}
    \end{center}
    { \hspace*{\fill} \\}
    

    % Add a bibliography block to the postdoc
    
    
    
    \end{document}
